\documentclass[a4paper,12pt]{article} 

\usepackage[a4paper, top=3cm, bottom=3cm, left=2cm, right=2cm]{geometry}
\usepackage{amsmath,amsthm,amssymb}
\usepackage{cmap}					% поиск в PDF
\usepackage[utf8]{inputenc}			% кодировка исходного текста
\usepackage[english]{babel}     	% локализация и переносы
\usepackage{csquotes}
\usepackage{multicol}
\usepackage{tikz}
\usetikzlibrary{positioning}

\bibliographystyle{abbrv}

\usepackage[ruled,linesnumbered]{algorithm2e}

\newtheorem{claim}{Claim}

\newcommand{\cq}[1]{\ensuremath{\mathsf{CQ}_{#1}}}
\newcommand{\dbi}{\ensuremath{\mathbf{I}}}
\newcommand{\bigo}{\ensuremath{\mathcal{O}}}
\newcommand{\query}[3]{\ensuremath{{#1}({#2})\:{:}{-}\:{#3}}}
\newcommand{\rr}{\mathsf{RR}}
\newcommand{\ranf}{\mathsf{RANF}}
\newcommand{\srnf}{\mathsf{SRNF}}
\renewcommand{\phi}{\varphi}

\begin{document}

\begin{center}
{\LARGE\bfseries Foundations of Databases}\\[3mm]

{\Large Coursework 3}\\[5mm]

Lorenz Leutgeb\\\texttt{lorenz.leutgeb@stud-inf.unibz.it}\\[2mm]
Anastassiya Pustozerova\\\texttt{anastassiya.pustozerova@stud-inf.unibz.it}
\end{center}

\section{Evaluation of Special Types of Conjunctive Graph Queries}

We let $n = |\dbi|$.

\paragraph{Circle Query}{
Algorithm \ref{alg:circle} computes answers for $C_k$. The join in line 3 joins two relations of at most rows $n$ and results a relation of at most $n$ rows. This requires $\bigo(n \log n)$ steps. Note that the join in line 3 is executed $k - 1$ times, since it is inside a loop. The selection and projection in line 5 can be achieved by a single scan over $R$, thus are in $\bigo(n)$. We arrive at an overall time complexity in $\bigo(k \cdot n \cdot \log n)$ or rather $\bigo(k \cdot |\dbi| \cdot \log |\dbi|)$.
\begin{algorithm}
	\KwIn{A positive non-zero integer $k$, a database instance (implicitly).}
	\KwOut{A relation that contains all nodes that lie on a circle of length at least $k$.}
	$R := \rho_{start \: \leftarrow\: 1, end\: \leftarrow\: 2}(edge)$\\
	\ForEach{$i \in \{1, \ldots, k - 1\}$}{
		$R := \rho_{end \:\leftarrow\:2} (\pi_{start, 2}(R \bowtie_{end = 1} edge))$\\
	}
	\Return{$\pi_{R.end}(\sigma_{R.start = R.end}R)$}
	\caption{$\mathsf{EvalCircle}(k)$}
	\label{alg:circle}
\end{algorithm}
}

\paragraph{Star Query}{
We exploit the fact that the query does not require $z_1, \ldots, z_k$ to be pairwise different, so $\alpha(z_1) = \cdots = \alpha(z_k)$ where all $z_i$ ($1 \leq i \leq k$) map to the same successor of $\alpha(x)$ will be an assignment for an answer to $S_k$. Time complexity of Algorithm \ref{alg:star} is in $\bigo(|\dbi|)$, since a single scan over all edges suffices.
\begin{algorithm}
\KwIn{A positive non-zero integer $k$, a database instance (implicitly).}
\KwOut{A relation that contains all nodes that are in the center of a star of at least $k$ nodes.}
\Return{$\Pi_{1}(edge)$}
\caption{$\mathsf{EvalStar}(k)$}
\label{alg:star}
\end{algorithm}
}

\paragraph{Spider-Web Query}{
Consider Algorithm \ref{alg:web}: The time complexity of line 1 is in $\bigo(n \cdot \log n)$ and the analysis for lines 2-5 is similar to the $\mathsf{EvalCircle}$. We arrive at a time complexity in $\bigo(k \cdot n \cdot \log n)$ for the algorithm.
\begin{algorithm}
	\KwIn{A positive non-zero integer $k$, a database instance (implicitly).}
	\KwOut{A relation that contains all nodes that are in the center of a spider web of size at least $k$.}
	$R := \rho_{center \: \leftarrow\: 1, start\: \leftarrow\: 2}(edge) \bowtie \rho_{start\: \leftarrow\: 1, end\: \leftarrow\: 2}(edge)$\\
	\ForEach{$i \in \{1, \ldots, k - 1\}$}{
		$R := \rho_{end \:\leftarrow\:2} (\pi_{center, start, 2}(R \bowtie_{end = 1} edge))$\\
	}
	\Return{$\pi_{R.center}(\sigma_{R.start = R.end}R)$}
	\caption{$\mathsf{EvalSpiderWeb}(k)$}
	\label{alg:web}
\end{algorithm}
}

\paragraph{Clique Query}{The clique problem (for varying $k$) is known to be NP-complete (and a reduction from it is obvious). Time complexity of a guess-and-check algorithm is in $\bigo(n^k \cdot k^2)$ where $n^k$ is an upper bound for the number of possible cliques (subsets of $k$ nodes of the original graph) to guess and checking full connectedness for $k$ takes $k^2$ steps.}

\section{Evaluation of Conjunctive Queries with Unary Relation Symbols}

We address two classes of queries separately. First, we restate the evaluation problem for conjunctive queries restricted to Boolean queries.

\begin{center}
\textbf{Evaluation Problem for a single Boolean Conjunctive Query}\\[1mm]
\begin{tabular}{rp{14cm}}
Given: & A database instance $\dbi$, and a boolean conjunctive query $\query{Q}{}{L, M}$ (where $L$ are relational atoms and $M$ are atoms over built-ins). \\
Question: & Does there exist a variable assignment $\alpha$ for the variables of $Q$ such that $\alpha(L) \subseteq \dbi$ and $\alpha \models M$, or equivalenty, does  $\{()\} = Q(\dbi)$ hold?
\end{tabular}
\end{center}

\paragraph{Relational Queries}{We consider boolean relational queries of the form $\query{Q_n}{}{L}$ where all elements of $L$ are relational atoms of arity one and $n$ indicates the number of existentially quantified variables occurring in $L$. Algorithm \ref{alg:rue} solves \textbf{query evaluation for these queries in polynomial time}. The key insight is that computing the union of two sets $R_1$ and $R_2$ can be achieved in time $\bigo(|R_1| \cdot |R_2|)$ (not assuming that the sets can be accessed exploiting some order on the elements). The idea behind the algorithm is to start with the maximal set of answers possible, which is $adom(\dbi)^n$ (without an explicit construction) and then mimimize this set using the relational atoms in the body of the query. For every variable $x_i$ ($1 \leq i \leq n$), an intermediate relation $R_{x_i}$ is introduced and initialized to equal $adom(\dbi)$. Then for every relational atom in which one of the variables occurs $R(x_i)$ ($1 \leq i \leq n$) the corresponding intermediate relation $R_{x_i}$ is filtered by $R$ accordingly by intersecting the two sets.

\begin{algorithm}
\KwIn{A boolean relational conjunctive query with unary relational symbols $\query{Q_n}{}{L}$ where $L$ are relational atoms, a database instance $\dbi$.}
\KwOut{\enquote{yes} iff $\{()\} = Q(\dbi)$, otherwise \enquote{no}.}
Let $\bar{x} = (x_1, \ldots x_n)$ contain all variables of $Q$ in arbitrary order.\\
\ForEach{$i \in \{1, \ldots, n\}$}{
	$R_{x_i} := adom(\dbi)$
}
\ForEach{$R(t) \in L$}{
	\Switch{depending on the type of $t$}{
		\Case{$x_i$ with $1 \leq i \leq n$ (existentially quantified variable)}{
			$R_{x_i} := R_{x_i} \cap R(\dbi)$\\
			\If{$R_{x_i} = \emptyset$}{\Return{\enquote{no}, since $x_i$ cannot be assigned.}}
		}
		\Case{$c$ (constant)}{
			\If{$(c) \not \in R(\dbi)$}{\Return{\enquote{no}}}
		}
	}
}
\Return{\enquote{yes}}
\caption{$\mathsf{EvalRelUnary}(Q,\dbi)$}
\label{alg:rue}
\end{algorithm}

The loop in lines 2-4 takes time in $\bigo(n \cdot |adom(\dbi)|)$, since it copies the active domain of the database instance $n$ times. The loop in lines 6-19 takes time in $\bigo(|L| \cdot |adom(\dbi)|^2)$: For every relational atom in the body of the query it (a) computes an intersection, where the upper bound of the size of both intersected sets is the size of the active domain of the database instance, or (b) checks containment for constants which takes linear time in $\dbi$. We observe that the loop in lines 6-19 dominates the time-behavior of the algorithm and relax $adom(\dbi)$ to $|\dbi|$ and $|L|$ as well as $n$ to $|Q|$, which yields an overall bound of $\bigo(|Q| \cdot |\dbi|^2)$. If we assume $Q$ (respectively $\dbi$) of constant size, we arrive at  $\bigo(|\dbi|^2)$ (respectively $\bigo(|Q|)$).
}

\paragraph{General Conjunctive Queries}{We consider the larger class of general conjunctive queries of the form $\query{Q}{x_1, \dots, x_n}{L,M}$ where all elements of $L$ are relational atoms of arity one and elements of $M$ are atoms over the built-in predicates $<$, $\leq$ and $\not =$.

We show that query evaluation for this class of queries is \textbf{NP-hard by reduction from} the well-known graph coloring problem.

\begin{center}
\textbf{Graph Coloring Problem}\\[1mm]
\begin{tabular}{rp{14cm}}
Given: & A directed graph $G = (V, E)$. \\
Question: & Does there exist a total labeling function $\alpha : V \rightarrow \{r, g, b\}$ such that for all $(v, u) \in E$ we have $\alpha(v) \not = \alpha(u)$.
\end{tabular}
\end{center}

Let $\dbi_{3} = \{ C(r), C(g), C(b) \}$, and take $V$ as a set of variables.\footnote{More formally, one may introduce $\mathcal{V}$ as an arbitrary infinite set of variable symbols and $\tilde{\cdot} : V \rightarrow \mathcal{V}$ as an injective function that maps a vertex $v \in V$ to an arbitrary variable symbol in $\mathcal{V}$.} Further, let $\query{Q_G}{}{L,M}$ be a (boolean) general conjunctive query where $L = \{ C(v) \mid v \in V \}$ and $M = \{ v \not = u \mid (v, u) \in E\}$.

\begin{claim}
For a given directed graph $G$, the answer to the Graph Coloring Problem is \enquote{yes} if-and-only-if the answer to the Evaluation Problem for a single Boolean Conjunctive Query is \enquote{yes} for $Q_G$.
\end{claim}

\begin{proof}
To see the if-direction: Let $\alpha$ be a labeling for $G$ according to the requirements of the Graph Coloring Problem by assumption. Then we need to show that $\bar{c} = ()$ is an answer tuple for $Q_G$ over $\dbi_3$ with the assignment $\alpha$: Vacuously $\bar{c} = \alpha(())$ (boolean query), $\alpha(v) \in \dbi_3(C)$ follows from the definition of $\alpha$ and the definition of $\dbi_3$ since the range of $\alpha$ is the same as $\dbi_3(C)$, and finally $\alpha \models M$ since the property imposed by the definition of the Graph Coloring Problem and the inequalities in $M$ coincide.

To see the only-if direction: Let $\bar{c} = ()$ be an answer tuple for $Q_G$ over $\dbi_3$ with the assignment $\alpha$ by assumption. Then we show that $\alpha$ is a labeling for $G$ as required by the Graph Coloring Problem: By $\alpha(v) \in \dbi_3(C)$ we have that the domain of $\alpha$ is at most $\{r, g, b\}$ and the inequalities in $M$ guarantee the required properties for $\alpha$.
\end{proof}

\section{Containment of Relational Conjunctive Queries without Self Joins}

Given two relational conjunctive queries without self joins $Q(\bar{x})$ and $Q'(\bar{x'})$ and the question $Q \stackrel{?}{\sqsubseteq} Q'$, our idea is to implement an algorithm that checks whether it is possible to find a query homomorphism $h : \mathcal{T}(Q') \rightarrow \mathcal{T}(Q)$ . By the Chandra-Merlin theorem $h$ exists if-and-only-if $Q \sqsubseteq Q'$ (note that we consider queries without built-ins).

In the lecture we saw two different definitions for query homomorphisms. The more strict one is defined in the slides (Part 4: Conjunctive Queries, slide 26). In the lecture we discovered a more general definition, where they key difference is how distinguished variables are handled: By the definition on the slides, distinguished variables must be mapped to themselves, i.e.\ $h(x'_i) = x'_i$ where $\bar{x'} = (x'_1, \ldots, x'_n)$ and $1 \leq i \leq n$, while for the more general definition, distinguished variables must agree under $h$, i.e.\ $h(\bar{x'}) = h(\bar{x})$.

Algorithm \ref{alg:homstrict} decides the existence of a homomorphism according to the stricter definition (see lines 16-21).

\begin{algorithm}
\KwIn{Two relational queries without self join $\query{Q}{\bar{x}}{L}$ and $\query{Q'}{\bar{x'}}{L'}$.}
\KwOut{\enquote{yes} iff $Q \sqsubseteq Q'$, otherwise \enquote{no}.}
\ForEach{relational atom $R'(t'_1, \ldots, t'_n) \in L'$}{
$f := \bot$\\
\ForEach{relational atom $R(t_1, \ldots t_m) \in L$}{
\If{$R \not = R'$ or $n \not = m$}{\textbf{continue}}
$f := \top$\\
\ForEach{$i \in \{1, \ldots, n\}$}{
\Switch{on the type of $t'_i$}{
	\Case{constant}{
		\If{$t_i$ is a constant and $t_i = t'_i$}{\textbf{continue}}
		\Return{\enquote{no}}, since some constant would not be mapped to itself, violating $h(c) = c$.
	}
	\Case{distinguished variable in $Q'$}{
		\If{$t_i$ is a distinguished variable in $Q$ and $t_i = t'_i$}{\textbf{continue}}
		\Return{\enquote{no}}, since some distinguished variable would not be mapped to itself, violating $h(x) = x$.
	}
	\Case{non-distinguished variable}{
		// $t'_i$ may be mapped to other variables or constants.\\
		\textbf{continue}
	}
}
%\Return{\enquote{no}}, since there cannot be a homomorphism from $Q'$ to $Q$.

%\If{$t'_i$ is a variable that is not a distinguished variable in $Q'$}{\textbf{continue}, since the non-distinguished variable can mapped to other variables or constants.}
%\If{$t'_i$ is a constant and $t_i$ is a constant and $t_i \not = t'_i$}{\Return{\enquote{no}}, since for any homomorphism $h(c) = c$.}
%\If{$t'_i$ is a constant and $t_i$ is not a constant}{\Return{\enquote{no}}, since for any homomorphism $h(c) = c$.}
%\If{$t'_i$ is a distinguished variable and $t[i]$ is a distinguished variable and $t_i \not = t'_i$}{\Return{\enquote{no}}, since distinguished variables must map to themselves.}}
}
\textbf{break}\\
}
\If{$f = \bot$}{\Return{\enquote{no}}, since there is no relational atom over the relation $R$ in $L$, violating $h(L') \subseteq L$.}
}
\Return{\enquote{yes}}
\caption{$\mathsf{NoSelfJoinContained}(Q, Q')$}
\label{alg:homstrict}
\end{algorithm}

For the more general definition, we want to reduce the decision problem of the existence of a homomorphism $h : \mathcal{T}(Q') \rightarrow \mathcal{T}(Q)$ to term matching. Intuitively, matchers and query homomorphisms coincide.

\begin{algorithm}
\KwIn{Two relational queries without self join $\query{Q}{\bar{x}}{L}$ and $\query{Q'}{\bar{x'}}{L'}$.}
\KwOut{\enquote{yes} iff $Q \sqsubseteq Q'$, otherwise \enquote{no}.}
\If{$|\bar{x}| \neq |\bar{x'}|$}{\Return{\enquote{no}}}
Let $\bar{x} = (x_1, \ldots , x_m)$ and $\bar{x'} = (x'_1, \ldots, x'_m)$.\\
$\theta := \varepsilon$ (a substitution/mapping from variables to terms also called \enquote{matcher})\\
$\mathcal{M} := \{ x_i \approx x'_i \mid 1 \leq i \leq m \}$ (a matching problem)\\

\ForEach{relational atom $R(t'_1, \ldots, t'_n) \in L'$}{
	\If{$R(t_1, \ldots, t_n) \not \in L$}{\Return{\enquote{no}}}
	$\mathcal{M} := \mathcal{M} \cup \{ t'_1 \approx t_1, \ldots, t'_n \approx t_n \}$ where $R(t_1, \ldots, t_n) \in L$\\
}

\While{$\mathcal{M} \not = \emptyset$}{
	Select an equation $s \approx t$ from $\mathcal{M}$.\\
	$\mathcal{M} := \mathcal{M} \setminus \{ s \approx t \}$\\
	\Switch{depending on the structure of $s \approx t$}{
		\Case{$x \approx x$ (same variable on both sides)}{
		nothing}
		\Case{$c \approx d$ where $c \neq d$ (different constants)}{\Return{\enquote{no}, analogously to decomposition of different functions in unification.}}
		\Case{$x \approx t$ or $t \approx x$ (variable on one side and term on the other side)}{
			$\theta := \theta\{x \mapsto t\}$ to extend the matcher and apply the substitution to existing mappings.\\
			$\mathcal{M} := \mathcal{M}\{x \mapsto t\}$ to apply the substitution to the remaining matching problem.
		}
	}
}
\Return{\enquote{yes}}
\caption{$\mathsf{NoSelfJoinContained}(Q, Q')$}
\label{alg:homgen}
\end{algorithm}

Line 18 depends on the definition of substitutions to be used. Commonly substitutions are defined to map variables to themselves \enquote{by default}. If this is not the case, then $x \mapsto x$ must be added to $\theta$ here.

% Comment why we require distinguished variables to map to themselves. More general definition of a homomorphism is possible and we saw one in the lecture, but the one on the slides is stricter.
\end{document}