\documentclass[a4paper,12pt]{article} 

\usepackage[a4paper, top=3cm, bottom=3cm, left=2cm, right=2cm]{geometry}
\usepackage{amsmath,amsthm,amssymb}
\usepackage{cmap}					% поиск в PDF
\usepackage[utf8]{inputenc}			% кодировка исходного текста
\usepackage[english]{babel}     	% локализация и переносы
\usepackage{csquotes}
\usepackage{tikz}
\usetikzlibrary{positioning}

\bibliographystyle{abbrv}

\usepackage[ruled,linesnumbered]{algorithm2e}

\newtheorem{claim}{Claim}

\newcommand{\cq}[1]{\ensuremath{\mathsf{CQ}_{#1}}}
\newcommand{\query}[3]{\ensuremath{{#1}({#2})\:{:}{-}\:{#3}}}
\newcommand{\rr}{\mathsf{RR}}
\newcommand{\ranf}{\mathsf{RANF}}
\newcommand{\srnf}{\mathsf{SRNF}}
\renewcommand{\phi}{\varphi}

\begin{document}

\begin{center}
{\LARGE\bfseries Foundations of Databases}\\[3mm]

{\Large Coursework 2}\\[5mm]

Lorenz Leutgeb\\\texttt{lorenz.leutgeb@stud-inf.unibz.it}\\[2mm]
Anastassiya Pustozerova\\\texttt{anastassiya.pustozerova@stud-inf.unibz.it}
\end{center}

\section{Safety of Positive Queries}

\begin{claim}
The safety of simple positive queries is decidable.
\end{claim}

\begin{proof}[Sketch of Proof] We specify an algorithm. Similar to the approach presented in the lecture we first translate $\phi$ into a normal form to allow for a concise rule-based formulation.

We choose (a subset of) SRNF as our normal form, and re-use the algorithm given in \cite[Section 5.4, p.\ 83]{alice} to translate a simple positive formula to SRNF. We call it $\srnf$. Note that some rules of this algorithm will never be applicable to simple positive formulae: universal quantification, implication and negation are not included in the definition of simple positive formulae. This hints why we only need to consider a subset of SRNF, intuitively \emph{simple positive SRNF}. Also note that the rewrite rules preserve equivalence and therefore $\phi$ is safe if and only-if $\srnf(\phi)$ is safe.

From here, we can use $\srnf$ and the algorithm below (adapted from \cite[Algorithm 5.4.3, p.\ 84]{alice} to decide safety of formulae in simple positive SRNF) to decide safety of simple positive formulae: A simple positive formula $\phi$ is safe if and only-if $\rr(\srnf(\phi))$ returns the set of all free variables of $\phi$.

\begin{algorithm}
  \KwIn{A formula $\phi$ in simple positive SRNF.}
  \KwOut{The subset of free variables of $\phi$ that are range-restricted, or $\bot$, indicating that a quantified variable is not range restricted.}
  \Switch{$\phi$}{
  	\uCase{$R(t_1,...,t_n)$}{\Return{the set of variables from $t_1,\dots,t_n$}}
  	\uCase{$\psi_1 \land \psi_2$}{\Return{$\rr(\psi_1) \cup \rr(\psi_2)$}}
  	\uCase{$\psi_1 \lor \psi_2$}{\Return{$\rr(\psi_1) \cap \rr(\psi_2)$}}
  	\uCase{$\exists{x} \: \psi$}{
  		\uIf{$x \in \rr(\psi)$}{
  			\Return{$\rr(\psi) \setminus \{x\}$}
  		}
  		\Else{
  			\Return{$\bot$ ~ ~ ~ \textnormal{Note that $\bot \cup S = \bot \cap S = \bot - S = \bot - S = \bot$ for any set $S$.}}
  		}
  	}
  }
  \caption{$\rr(\phi)$}
\end{algorithm}

Termination can be shown by induction over the structure of $\phi$. Or top-down since the number of subformulae of $\phi$ is finite and $\rr$ is recursively called on these subformulae, with termination in case of atoms (see case $R(t_1,...,t_n)$ above).
\end{proof}

\begin{claim}
Every safe simple positive query is domain independent. 
\end{claim}

\begin{proof}[Sketch of Proof]
By the safe range theorem all safe range queries are domain independent. Since we consider positive simple SRNF in the above Sketch of Proof it is applicable here. However, we elaborate by specifying an algorithm that translates a formula in simple positive SRNF into simple positive RANF (a subset of relational algebra normal form, RANF). In a further step we show that these formulae can be translated into corresponding relational algebra expressions for which safety is known.

The trasnlation to simple positive RANF given is similar to \cite[Algorithm 5.4.7, p.\ 88]{alice}. Note that we do not consider $\land$ and $\lor$ in their polyadic form but only their binary form, and we also do not consider existential quantification over a sequence of variables (of the form $\exists{x_1, \dots, x_n}$ or $\exists{\vec{x}}$) but only for single variables, i.e.\ of the form $\exists{x}$. We trade a more concise notation of rules for an increase in the number of applications of these rules.

\begin{algorithm}
  \KwIn{A formula $\phi$ in simple positive SRNF.}
  \KwOut{A formula $\phi'$ in RANF, equivalent to $\phi$.}
  \While{some subformula $\psi$ (with its conjuncts possibly reordered) of $\phi$ satisfies the premise of R1 or R2}{
  \Switch{depending on premises of R1 or R2}{
  	\uCase{$\psi = \psi_1 \land \cdots \land \psi_2 \land \exists{x} \: \xi$ where $\rr(\psi)$ is the set of free variables in $\psi$ but $rr(\xi)$ is not the set of free variables in $\xi$}{Replace $\psi$ by $\srnf(\psi)$\\$\phi := $ result of replacing $\psi$ by $\psi'$ in $\phi$} %TODO
  	\uCase{R2 applicable}{Push-into-quantifier} %TODO
  }}
  \caption{$\ranf(\phi)$}
\end{algorithm}
\end{proof}

\section{Unions of Conjunctive Queries}

\begin{claim}
Adding union to simple conjunctive queries strictly increases expressivity of the resulting language.
\end{claim}

\begin{proof}
We show by contradiction that there is a union of simple conjunctive queries $Q(x)$ that cannot be expressed by a simple conjunctive query. Let
\begin{align*}
\query{Q}{x}{p(x)} && \query{Q}{x}{r(x)}
\end{align*}
and assume that the simple conjunctive query $\query{Q'}{x}{L'}$ is equivalent to $Q(x)$ towards a contradiction.

Since $Q(x)$ and $Q(x)$ are equivalent by assumption, $Q(\textbf{I}) = Q'(\textbf{I})$ for any database instance $\textbf{I}$ by definition of query evaluation. In the following we consider the instance $\textbf{I}_Q = \{ r(c_{x_1}), p(c_{x_2}) \}$. Observe $\{ (c_{x_1}), (c_{x_2}) \} = Q(\textbf{I}_Q) = Q'(\textbf{I}_Q)$. By definition of answer tuples there are $\alpha_1$, $\alpha_2$ such that
\begin{align*}
\alpha_1(x) &= c_{x_1} &
\alpha_1(L') &\subseteq \textbf{I}_Q \\
\alpha_2(x) &= c_{x_2} &
\alpha_2(L') &\subseteq \textbf{I}_Q
\end{align*}
From the above properties of $\alpha_1$ we have that $p(x) \in L'$ and analogously using the above properties of $\alpha_2$ we have $r(x) \in L'$. Together, we have $\{ r(x), p(x) \} \subseteq L'$.

However, this yields a contradiction since $\{p(c_{x_1})\} \subseteq \alpha_1(L') \not \subseteq \textbf{I}_Q$ and analogously $\{r(c_{x_2})\} \subseteq \alpha_2(L') \not \subseteq \textbf{I}_Q$. This closes our proof by showing the opposite of what was assumed, i.e.\ that there is no simple conjunctive query equivalent to $Q(x)$.
\end{proof}

\section{Classes of Conjunctive Queries}

$\cq{} \subseteq \cq{=}$ since \cq{} is the more specific subset of queries inside \cq{=} that do not have equality atoms in the body. The inclusion is obvious.

Next, we consider inclusions between \cq{}, \cq{rep}, \cq{const} and \cq{rep,const}.
$\cq{} subseteq \cq{rep} \subseteq \cq{rep,const}$ since \cq{} is the more specific subset of queries inside \cq{rep} that have no repetitions in the head, and \cq{rep} is the more specific subset of queries inside \cq{rep,const} that have no constants in the head. These inclusions are obvious.
$\cq{} \subseteq \cq{const} \subseteq \cq{rep,const}$ since \cq{} is the more specific subset of queries inside \cq{const} that have no constants in the head, and \cq{const} is the more specific subset of queries inside \cq{rep,const} that have no repetitions in the head. These inclusions are obvious.

To show $\cq{=} \subseteq \cq{rep,const}$ let $Q(\bar{x}) :- L, M$ where $Q(x) \in \cq{=}$ and $n$ be the number of (distinct) inequality atoms in $Q$. We use $M = \{ x_i = y_i \mid 1 \leq i \leq n\}$ to denote the set of equality atoms in $Q$. Without loss of generality we assume that for all equality atoms in $M$ the following holds: (a.) If one of the two terms occurring in the equality atom is a constant, then it is to the right of the equality symbol. (b.) If $x = y$ is in $M$ then $y = x$ is not in $M$ (some ordering on variables is respected). We transform $Q(x)$ into an equivalent query $Q'(x') :- L'$ where $Q'(x) \in \cq{rep,const}$ as follows: We take the set of relational atoms $L$ from $Q$ and translate it to a set of relational atoms $L'$ for $Q'$ by considering $M$. For every equality $x_i = y_i$ in $M$, the relational atoms in $L$ are rewritten by replacing $x_i$ with $y_i$. In a similar way, occurences of $x_i$ in the head of $Q$, are replaced by $y_i$ to yield the head of $Q'$. For example:

\begin{align}
Q(x,y,z) :- R(x,y), Q(y,z), x = y, z = c \\
Q'(x,x,c) :- R(x,x), Q(x,c)
\end{align}

Applying this process in reverse (eliminating repetition/constants in the head by introducing a new variable and an equality atom) shows how to translate a query in \cq{rep,const} to a query in \cq{=}, therefore $\cq{=} \supseteq \cq{rep,const}$.

Together with $\cq{rep} \subseteq \cq{rep,const}$ and $\cq{const} \subseteq \cq{rep,const}$ we have $\cq{=} \supset \cq{rep}$ and $\cq{=} \supset \cq{const}$.

3 Repetition does not really increase expressivity (duplicate query):
To remove repeated variables, introduce new variables and alias them using =.
$\cq{}    \subseteq \cq{rep}$
$\cq{rep} \subseteq \cq{}$

$\cq{const} \not \supseteq \cq{rep,const}$

$\cq{=} \not \subseteq \cq{rep}$ by counterexample: For the query $Q(x) :- x = c$ there is no equivalent query in \cq{rep}.

$\cq{=} \not \subseteq \cq{const}$ by counterexample: For the query $Q(x,y) :- x = y$ there is no equivalent query in \cq{const}.

$\cq{} \not \supseteq \cq{const}$ by counterexample: For the query $Q(c) :- R(y)$ there is no equivalent query in \cq{}. 

$\cq{} \not \supseteq \cq{rep}$ by counterexample: For the query $Q(x,x) :- R(x)$ there is no equivalent query in \cq{}.

$\cq{rep} \not \supseteq \cq{rep,const}$ by counterexample: For the query $Q(x,x,c) :- R(x)$ there is no equivalent query in \cq{}.

$\cq{const} \not \supseteq \cq{rep,const}$ by counterexample: For the query $Q(x,x,c) :- R(x)$ there is no equivalent query in \cq{}.

Remaining:
$\cq{} \not \supseteq \cq{=}$

We use the result $\cq{} = \cq{=}$.

\begin{center}
\begin{tabular}{|l|cccc|}
\hline
               &       \cq{=} &     \cq{rep} &   \cq{const} & \cq{rep,const} \\
\hline
\cq{}          & $\subsetneq$ & $\subsetneq$ & $\subsetneq$ &   $\subsetneq$ \\
\cq{=}         &              & $\supsetneq$ & $\supsetneq$ &              = \\
\cq{rep}       &              &              &              &   $\supsetneq$ \\
\cq{const}     &              &              &              &   $\supsetneq$ \\
\hline
\end{tabular}
\end{center}

\begin{center}
\begin{tikzpicture}
		\node (t)                          {\cq{=}, \cq{rep,const}};
		\node (l) [below left =1.5cm of t] {\cq{rep}};
		\node (r) [below right=1.5cm of t] {\cq{const}};
		\node (b) [below      =  3cm of t] {\cq{}};
		\draw (b) -> (l);
		\draw (b) -> (r);
		\draw (l) -> (t);
		\draw (r) -> (t);
\end{tikzpicture}
\end{center}

\bibliography{2}

\end{document}