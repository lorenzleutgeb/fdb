\documentclass[a4paper,12pt]{article} 

\usepackage[a4paper, top=3cm, bottom=3cm, left=2cm, right=2cm]{geometry}
\usepackage{amsmath,amsthm,amssymb}
\usepackage{cmap}					% поиск в PDF
\usepackage[utf8]{inputenc}			% кодировка исходного текста
\usepackage[english]{babel}     	% локализация и переносы
\usepackage{csquotes}
\usepackage{multicol}
\usepackage{tikz}
\usetikzlibrary{positioning}

\bibliographystyle{abbrv}

\usepackage[ruled,linesnumbered]{algorithm2e}

\newtheorem{claim}{Claim}
\newtheorem{prop}{Proposition}

\newcommand{\cq}[1]{\ensuremath{\mathsf{CQ}_{#1}}}
\newcommand{\dbi}{\ensuremath{\mathbf{I}}}
\newcommand{\bigo}{\ensuremath{\mathcal{O}}}
\newcommand{\query}[3]{\ensuremath{{#1}({#2})\:{:}{-}\:{#3}}}
\newcommand{\rr}{\mathsf{RR}}
\newcommand{\ranf}{\mathsf{RANF}}
\newcommand{\srnf}{\mathsf{SRNF}}
\renewcommand{\phi}{\varphi}

\begin{document}

\begin{center}
{\LARGE\bfseries Foundations of Databases}\\[3mm]

{\Large Coursework 4}\\[5mm]

Lorenz Leutgeb\\\texttt{lorenz.leutgeb@stud-inf.unibz.it}\\[2mm]
Cosimo Damiano Persia\\\texttt{cosimodamiano.persia@stud-inf.unibz.it}
\end{center}

\section{Containment of Selfjoin-free Conjunctive Queries with Built-ins}

This problem can be solved in polynomial time. We consider $\query{Q}{\bar{s}}{L, M} \stackrel{?}{\sqsubseteq} \query{Q'}{\bar{s'}}{L', M'}$ and assume that whenever $M \models s = t$, then $s = t$ are syntactically equal (see hint).

First, construct $\query{Q_0}{\bar{s_0}}{L}$ and $\query{Q'_0}{\bar{s'_0}}{L'}$ (where distinguished variables that only occur in the built-in part of the body are removed). Use the polynomial time algorithm from the previous question to check $Q_0 \stackrel{?}{\sqsubseteq} Q'_0$. If this containment does not hold, then the original containment does not hold as well.

Consequently, we treat the case where $Q_0 \sqsubseteq Q'_0$ holds. Since this inclusions holds, there is a query homomorphism $h : \mathcal{T}(Q'_0) \rightarrow \mathcal{T}(Q_0)$. %From here, we use the built-in atoms from $M, M'$ to extend $h$.

What remains to be checked is that $M \models h(M')$. Equivalently, we may check whether $M \cup \neg h(M)$\footnote{With $\neg \cdot : x \leq y \mapsto x > y, x < y \mapsto x \geq y, x = y \mapsto x \neq y, \ldots$ and $\neg A = \{ \neg x \mid x \in A\}$.} is unsatisfiable (cf.\ Deduction Theorem, SAT). Without loss of generality we assume that all built-ins in $\neg h(M)$ and $M'$ are either $>$, $\neq$, or $\geq$. Construct a graph with one set of nodes and two different sets/types of edges. The set of nodes corresponds to variables in $ \cup \neg h(M)$, and edges correspond to built-in atoms. Edge type 1 corresponds to $>$ and $\neq$, while edges of type 2 correspond to $\geq$.

If there is a cycle with edges of mixed type in the graph, then $M \models h(M')$, thus the original inclusion does not hold. Otherwise, the original inclusion holds.

% TODO: Correctness.

\section{Injective and Surjective Mappings}

\begin{itemize}
\item{Given $f : X \rightarrow Y$ and $g : Y \rightarrow Z$ injective functions.
%Since $f$ is injective, for all $x, x' \in X$ we have $f(x) = f(x')$ implies $x = x'$.
%Since $g$ is injective, for all $y, y' \in G$ we have $g(y) = g(y')$ implies $z = z'$.
Let $x, x' \in X$ and assume $g(f(x)) = g(f(x'))$. By injectivity of $g$ we have $f(x) = f(x')$, and by injectivity of $f$ we have that $x = x'$, hence, $g \circ f$ is injective.
%Since the range of $f$ is $Y$, we may choose $f(x) = y$ and $f(x') = y'$.
}
\item{Given $f : X \rightarrow Y$ and $g : Y \rightarrow Z$ surjective functions.
Let $z \in Z$. By surjectivity of $g$ there exists a $y \in Y$ such that $g(y) = z$. By surjectivity of $f$ there exists a $x in X$ such that $f(x) = y$. By replacing $f(x) = y$ in $g(y) = z$ we get $g(f(x)) = z$, thus $x$ has a preimage in $X$ with respect to $g \circ f$, hence, $g \circ f$ is surjective.
}
\item{
Given $g \circ f$ injective. Then both $g$ and $f$ are injective.

First we show $f$ injective: For any $x, x' \in X$ s.t.\ $x \neq x'$ we need to show $f(x) \neq f(x')$. This can be seen by contradiction, since $f(x) = f(x')$ would contradict injectivity of  $g \circ f$, hence $f$ injective.

To see that $g$ is injective, assume towards a contradiction that $g$ is not injective.  Let $x, x' \in X$ s.t.\ $x \neq x'$ and $f(x) = y$ as well as $f(x') = y'$. Since $f$ injective, we have $y \neq y'$. Without loss of generality $g(y) = g(y')$ since $g$ is not injective. However, we reach a contradiction, since $g(f(x)) = g(f(x'))$. Therefore, $g$ must be injective.
}
\item{
Given $g \circ f$ surjective. Then both $g$ and $f$ are surjective.

First we show that $g$ is surjective: Let $z \in Z$ be arbitrary and $x \in X$ such that $g(f(x)) = z$ (by surjectivity of $g \circ f$). Hence, we can find some $f(x) = y \in Y$ such that $g(y) = z$. So $g$ is surjective.

To show that $f$ is surjective, assume to the contrary that $f$ is not surjective. Then, there is $y \in Y$ such that there is no $x \in X$ with $f(x) = y$. This means that there is no $y \in Y$ such that $z \in Z$ with $g(y) = z$, which contradicts that $g$ is surjective (see previous proof).
}
\item{
Given a finite set $X$ and $f : X \rightarrow X$ surjective (injective). Then $f$ is injective (surjective), i.e.\ surjectivity and injectivity coincide with bijectivity.
(Let $X$ be a finite set, and $f: X \rightarrow X $ is a function. Then $f$ is injective iff $f$ is surjective.)
\begin{proof}
 First, suppose $f$ is injective. If S has $n$ elements, by our assumption, this means the image of $f$ has at least $n$ elements. But the image of $f$ is contained in S, so it has at most $n$ elements; so the image of $f$ contains exactly $n$ elements and is therefore the whole of S, i.e. $f$ is surjective.

 Now, suppose $f$ is surjective. So, for each $y \in S$, there is an $x \in S$ such that $y=f(x)$. We choose one such $x$ for each $y$ and define a function $g:X \rightarrow X$ so that $g(y)=x$. By construction, $f(g(y))=y$, so g must be injective, and hence, must be surjective by the above argument. So $g$ is a bijection, and f is a left inverse for $g$. But a left inverse for a bijection is also a right inverse, so this implies f is a bijection, and then an injection.
\end{proof}
}
\end{itemize}

\section{Minimisation of Conjunctive Queries}

% HINT: It will be necessary to make use of some facts about injective and surjective mappings on finite sets.

\begin{prop} Let $Q$ be a RCQ with $n$ atoms and $Q'$ be an equivalent RCQ with $m$ atoms where $m < n$. Then there exists a subquery $Q_0$ of $Q$ such that $Q_0$ has at most $m$ atoms in the body and $Q_0$ is equivalent to $Q$.
\end{prop}

\begin{proof}
the containments $Q \sqsubseteq Q^{'}$ and $Q^{'} \sqsubseteq Q$ hold. By the homomorphism theorem there exist an homomorphism $h : Q \rightarrow Q^{'} $ and $h^{'} : Q^{'} \rightarrow Q $. The composition of the two homomorphisms $\phi : h \circ h^{'}$ applied to L gives as output a new query $\query{Q}{\bar{s}}{L_0}$. $L_0$ has at most m atoms, and $\phi$ is an homomorphism since distinguished variables and constants are mapped to themselves, and $\phi (L) \subseteq L_0$.
Since $\phi$ is an homomorphism, then the inclusion $Q_0 \sqsubseteq Q$ holds. 
If we consider the identity homomorphism then also $Q \sqsubseteq Q_0$ holds. Then exist a query $Q_0$ that has at most $m$ atoms and it is equivalent to $Q$.

\end{proof}

\begin{prop} Let $Q$ be a RCQ with $n$ atoms and $Q'$ be an equivalent RCQ with $m$ atoms where $m < n$. Then there exists a subquery $Q_0$ of $Q$ such that $Q_0$ has at most $m$ atoms in the body and $Q_0$ is equivalent to $Q$.
Let $Q$ and $Q_0$ be two equivalent minimal RCQs. Then $Q$ and $Q_0$
are identical up to renaming of variables.
\end{prop}

\begin{proof}
By the homomorphism theorem there exist an homomorphism $h : Q \rightarrow Q^{'} $ and $h^{'} : Q^{'} \rightarrow Q $. The composition of the two homomorphisms $\phi : h \circ h^{'}$, is surjective and since the domain and the codomain are the same set then the function must also be surjective, then $\phi$ is bijective. From $\phi$ beign bijective, we can conclude that $h$ is injective. Analogously $\phi^{'} : h^{'} \circ h$ is a bijective mapping  and also $h^{'}$ is injective. Since $h$ and $h^{'}$ are  mappings from the same cardinality then $h$ and $h^{'}$ are surjective, and then bijective. Bijectivity of the two homomorphisms mean that the same numbers of variables and constants occur in the queries. Hence the homomorphisms are the renaming of variables and $Q$ and $Q^{'}$ are equal up to renaming of variables.


\end{proof}

\section{Containment of Unions of Conjunctive Queries}

By CQ-in-CQ we denote the containment problem for relational queries.

\begin{claim}[UCQ-in-CQ]
Given a union of conjunctive queries $Q = \bigcup_{i=1}^n Q_i$ and a conjunctive query $Q'$. The containment $Q \sqsubseteq Q'$ holds iff for all $1 \leq i \leq n$ the containment $Q_i \sqsubseteq Q$ holds.
\end{claim}

UCQ-in-CQ is NP-complete:

\begin{description}
\item{NP-hard} by reduction from CQ-in-CQ, which is the special case $n = 1$.
\item{in NP} since an algorithm that iterates over all $Q_j$ and returns a positive answer if $Q_i \sqsubseteq Q$ (which is an NP-complete problem) for all $i$ decides the problem in NP ($n$ times NP is still NP)
\end{description}

\begin{claim}[CQ-in-UCQ]
Given a conjunctive query $Q$ and a union of conjunctive queries $Q' = \bigcup_{j=1}^m Q'_j$. The containment $Q \sqsubseteq Q'$ holds iff for some $1 \leq j \leq m$ the containment $Q \sqsubseteq Q_j$ holds.
\end{claim}

CQ-in-UCQ is NP-complete:

\begin{description}
\item{NP-hard} by reduction from CQ-in-CQ, which is the special case $m = 1$
\item{in NP} since an algorithm that iterates over all $Q_j$ and returns a positive answer if $Q_j \not \sqsubseteq Q$ (which is an NP-complete problem) decides the problem in NP ($m$ times NP is still NP)
\end{description}

\begin{claim}[UCQ-in-UCQ]
Given a union of conjunctive queries $Q = \bigcup_{i=1}^n Q_i$ and $Q' = \bigcup_{j=1}^m Q'_j$. The containment $Q \sqsubseteq Q'$ holds iff $Q_i \sqsubseteq Q_j$ for all $1 \leq i \leq n$ and for some $1 \leq j \leq m$ the containment $Q_i \sqsubseteq Q_j$ holds.
\end{claim}

UCQ-in-UCQ is NP-complete:

\begin{description}
\item{NP-hard} by reduction from CQ-in-CQ, which is the special case $n = m = 1$
\item{NP} since as above, we may run CQ-in-CQ $n \cdot m$ times (in the worst case).
\end{description}
\end{document}